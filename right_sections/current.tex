\cvsection{{\Large Currently Working On}}
	\cvachievement{\faBook}{Came first in the ML4SCI Hackathon}{\href{https://github.com/ML4SCI/ML4SCIHackathon/tree/main/HiggsBosonClassificationChallenge}{The Higgs Challenge}}
	
	\hspace*{8.2mm}\faCalendar Nov 2021 - Jan 2022
	\vspace{-2mm}
	\begin{justify}
		I, with my friend, participated in this competition. We were given 6 problems and were supposed to solve one (or more) of them using our Machine Learning skills. We went with the Higgs Challenge. We were given a dataset of 11 million data points and were expected to predict the presence of Higgs boson. We used an ensemble model which consisted of 5 neural network architectures and 1 XGBoost architecture. We were able to achieve an area of 0.88 under the ROC Curve.

		\vspace{3mm}

		\noindent GitHub Repository: \href{https://github.com/PeithonKing/ML_comp}{\textbf{PeithonKing/ML\_comp}}
	\end{justify}